%

% 1.5-facher Zeilenabstand:
\usepackage[onehalfspacing]{setspace} 

\usepackage[utf8]{inputenc}
\usepackage[ngerman]{babel}
% Durchgehende Nummerierung von Abbildungen über
% Chapter hinaus
\usepackage{chngcntr}
\counterwithout{figure}{chapter}
%
% Kommentare mit \comment{}-Umgebung
\usepackage{verbatim}
%
\usepackage[%
    backend=biber,
    style=numeric,
    %style=alphabetic,
    %citestyle=alphabetic-verb,
    %citestyle=numeric
    %sorting=ynt    % Sortierung nach Name des Autors
    sorting=none    % Sortierung nach Auftreten
]{biblatex}
%\addbibresource{data/bibliography.bib}
\addbibresource{Bachelorarbeit.bib}

\usepackage{fancyvrb}
\usepackage{listings}

\lstset{literate=%
  {Ö}{{\"O}}1
  {Ä}{{\"A}}1
  {Ü}{{\"U}}1
  {ß}{{\ss}}2
  {ü}{{\"u}}1
  {ä}{{\"a}}1
  {ö}{{\"o}}1
}

\usepackage{color}

\lstdefinelanguage{JavaScript} {
	morekeywords={
		break,const,continue,delete,do,while,export,for,in,function,
		if,else,import,in,instanceOf,label,let,new,return,switch,this,
		throw,try,catch,typeof,var,void,with,yield
	},
	sensitive=false,
	morecomment=[l]{//},
	morecomment=[s]{/*}{*/},
	morestring=[b]",
	morestring=[d]'
}

\lstdefinelanguage{swift}
{
  morekeywords={
    func,if,then,else,for,in,while,do,switch,case,default,where,break,continue,fallthrough,return,
    typealias,struct,class,enum,protocol,var,func,let,get,set,willSet,didSet,inout,init,deinit,extension,
    subscript,prefix,operator,infix,postfix,precedence,associativity,left,right,none,convenience,dynamic,
    final,lazy,mutating,nonmutating,optional,override,required,static,unowned,safe,weak,internal,
    private,public,is,as,self,unsafe,dynamicType,true,false,nil,Type,Protocol,
  },
  morecomment=[l]{//}, % l is for line comment
  morecomment=[s]{/*}{*/}, % s is for start and end delimiter
  morestring=[b]" % defines that strings are enclosed in double quotes
}
%
% https://github.com/mwhittaker/texmf/blob/master/tex/latex/code/java.sty
%
\lstdefinelanguage{java}{%
  % Basic settings
  basicstyle=\smaller\ttfamily,
  tabsize=4,
  %frame=single,
  showstringspaces=false,
  mathescape=true,
  breaklines=true,
  numbers=left,
  % Keywords, strings, and comments
  keywords={%
    abstract, continue, for, new, switch, assert, default, goto, package,
    synchronized, boolean, do, if, private, this, break, double, implements,
    protected, throw, byte, else, import, public, throws, case, enum,
    instanceof, return, transient, catch, extends, int, short, try, char,
    final, interface, static, void, class, finally, long, strictfp, volatile,
    const, float, native, super, while
  },
  keywords=[2]{%
  },
  morestring=[b]",
  morestring=[b]',
  morecomment=[l]{//},
  morecomment=[s]{/*}{*/},
  % Colors and style
  %backgroundcolor=\color{BackgroundYellow},
  keywordstyle=\color{blue},
  keywordstyle=[2]\color{DarkOrchid},
  commentstyle=\color{ForestGreen},
  stringstyle=\color{Red},
  numberstyle=\color{SolarizedGrey}
}

% environment
\lstnewenvironment{Java}[1][]{%
  \lstset{language=java, linewidth=80\charwidth, #1}
  \captionsetup{options=JAVA}
}{}


% inline definition.
\newcommand*{\java}{\lstinline[{language=java}]}

% input listing definition
\newcommand{\inputjava}[2][ ]{%
  {%
  \captionsetup{options=JAVA}
  \lstset{linewidth=80\charwidth}
  \lstinputlisting[language=java,#1]{#2}
  }
}

\lstset{
	basicstyle=\footnotesize\ttfamily,
	showstringspaces=false,
	keywordstyle=\ttfamily\bfseries,
	identifierstyle=\ttfamily,
	stringstyle=\ttfamily,
	xleftmargin=5pt,
	xrightmargin=5pt,
	aboveskip=\bigskipamount,
	belowskip=\bigskipamount
}

\usepackage{minted}
\usepackage[german]{fancyref}
\usepackage{enumitem}
\usepackage{array}
\usepackage{graphicx}
\usepackage{multicol}
%\usepackage{pgf-umlsd}
%\usepackage[pict2e]{struktex}
\usepackage{tikz}
%\usetikzlibrary{arrows,shadows}
%\usepackage{csvsimple}
%\usepackage{pgfplots}
\usepackage{filecontents}
%\usepackage{../daten/tikz-uml}

% Anführungsstriche mit \enquote{}-Umgebung
\usepackage[autostyle]{csquotes} 

% =============================
\usepackage[colorinlistoftodos,prependcaption,textsize=tiny]{todonotes}

% Hyperlinks in PDF-Version des Dokumentes. Option sagt, dass keine roten Boxen erzeugt werden sollen.
\usepackage[pdfborder={0 0 0}]{hyperref} 

% Todo commands
% https://mirror.hmc.edu/ctan/macros/latex/contrib/todonotes/todonotes.pdf
\newcommand{\insertMore}[1]{\todo[inline, color=green!40]{#1 ergänzen}}
\newcommand{\insertRef}[1]{\todo[color=blue!40]{#1 (Referenz fehlt)}}
\newcommand{\importantTodo}[1]{\todo[color=red!40]{#1 (Wichtig!)}}
