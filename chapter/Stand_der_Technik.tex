\chapter{Stand der Technik}
% 1 Seite
Smartglasses sind wie Brillen getragene Displays. Es gibt grundsätzlich zwei verschiedene Arten von Smartglasses: \cite{Schweizer2014SmartApplications, Ernst2016TheSubstitutability}
\begin{itemize}
    \item[]\emph{Virtual Reality (VR):} 
        Ziel ist es, die real wahrgenommene Welt vollständig durch eine virtuelle Welt zu ersetzen. Handlungen (z.B. Bewegungen) des Nutzers beeinflussen dabei die dargestellte virtuelle Welt. Ein Beispiel für eine VR-Brille ist die Oculus Rift.
    
    \item[] \emph{Augmented Reality (AR):} 
        Die reale Welt des Nutzers wird erweitert um virtuelle (visuelle, auditive oder haptische) Aspekte. Die reale Welt wird also mit einer virtuellen Realität kombiniert. Interaktion mit dem Gerät führt zu einer Anpassung der hinzugefügten virtuellen Welt. Beispiele für  AR-Brillen sind die Google Glass und die hmt-1 von Realwear. \todo{hier auf hmt1 eingehen?}
\end{itemize}

Im Feld der Augmented Reality 


\begin{comment}





\end{comment}
%
% - - - - - - - - - - - - - - - - - - - - - - - - 
%
\section{Smartglasses im beruflichen Umfeld}
% 3 Seiten
%
% - - - - - - - - - - - - - - - - - - - - - - - - 
%
\section{Entwicklungsstand Smartglasses}
% 3 Seiten
%
% - - - - - - - - - - - - - - - - - - - - - - - - 
%
\section{Vergleich verschiedener Smartglasses}
% 2 Seiten