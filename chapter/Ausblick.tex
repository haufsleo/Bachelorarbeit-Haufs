%
%
%
% - - - - - Ausblick - - - - - - - - 
%
%
%
\chapter{Ausblick}
\label{ch:Ausblick}
%
Die Erarbeitung im Stand der Technik (Kapitel \ref{sec:Stand_der_Technik}), die Anforderungsanalyse (Kapitel \ref{ch:Anforderungsanalyse}) sowie die Evaluation eines Prototypen für die multimediale Nutzung von Smartglasses in der ZSVA (Kapitel \ref{ch:Evaluation_des_Prototyps}) ergab weitere Möglichkeiten von Smartglasses in der ZSVA. Insbesondere die Nutzung \emph{echter} AR-Brillen wie der Microsoft Hololens ermöglichen deutlich größere Funktionalitäten und Eingriffe in die Realität der Nutzer. So könnten dreidimensionale Objekte ins Sichtfeld der Nutzer eingeblendet werden. In der ZSVA könnten diese \enquote{Hologramme} dazu genutzt werden, um detaillierte Zusatzinformationen zu Sterilgütern darzustellen. Dieser Einsatz müsste in einer weiteren wissenschaftlichen Arbeit erörtert und in einer Studie mit einem entsprechenden Prototypen evaluiert werden.

Der im Rahmen dieser Arbeit entwickelte Prototyp liefert lediglich die Basisfunktionalität einer Anwendung in der ZSVA. Eine wirkliche Anwendung müsste, wie in der Nutzerbefragung herausgefunden, an das User-Interface der ZSVA-Anwendungen angepasst werden. Die Möglichkeiten des Prototypen können durch die in der Nutzerbefragung erhaltenen Anregungen erweitert werden. So könnte zur Basisfunktionalität der Anwendung einige Funktionen des Videoplayers hinzugefügt werden. Sinnvoll für eine wirkliche Anwendung für die ZSVA ist sicherlich die Speicherung von Videos auf einem lokalen Server. So könnten aufgenommene Videos und Bilder anderen Mitarbeitern zur Verfügung gestellt werden. Eine Datenbankanbindung wäre ebenfalls sinnvoll. 

Denkbar ist auch die Entwicklung einer Anwendung, um Slideshows und Kapitelmarken für Videos zu erstellen, da bislang diese Dateien über JSON-Dateien erstellt werden müssen.
%
\insertMore{Ausblick}
