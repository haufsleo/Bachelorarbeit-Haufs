\chapter{Fazit}
%
Im Rahmen dieser Arbeit wurden verschiedene Ansätze zur Entwicklung mobiler Applikationen miteinander verglichen. Im Schwerpunkt der Arbeit wurde das Hybride Entwicklungsframework Ionic in der neuesten Version 2 vorgestellt. Als frei verfügbares Framework bietet es Entwicklern die Möglichkeit, ohne hohe finanzielle Zusatzbelastung mobile Anwendungen zu entwickeln. Als auf dem Framework Cordova aufbauendes Framework nutzt Ionic dabei das in seinen Grundzügen vorgestellte Framework Angular, um auf elegante Art und Weise Single-Page-Webanwendungen zu schreiben. Eingebettet in eine Hybride Wrapper-App werden diese Single-Page-Apps mit Schnittstellen versorgt, die native Funktionalitäten ermöglichen. 

Der große Vorteil von hybriden Ionic-Apps ist, dass diese über Pluginmodule erweitert werden können. So können theoretisch alle Funktionalitäten nativer Apps erreicht werden.

Am Beispiel einer sehr speziellen nativen iOS-Funktionalität wurde dies im Rahmen dieser Arbeit vorgestellt. Die Kommunikation zwischen einer hybriden Ionic-App und einer nativen in Swift geschriebenen Apple Watch-App ist dank eines Plugins prinzipiell möglich, zeigt zugleich jedoch auch aufgrund der Implementation des Plugins eine Grenze auf. So ist der mögliche Speicher, der für den Datenaustausch verwendet werden kann, sehr begrenzt. Wenn Entwickler native Funktionalität in einer Ionic-App voll nutzen wollen müssen entweder die entsprechenden Plugins zur Verfügung stehen oder selbst entwickelt werden.