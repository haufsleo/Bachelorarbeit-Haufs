%
%
%
% - - - - - Fazit - - - - - - - - 
%
%
%
\chapter{Fazit}
\label{ch:Fazit}

% 3 Seiten
Im Rahmen dieser Arbeit wurden Smartglasses und insbesondere die Möglichkeiten von Assisted Reality Smartglasses im Bereich der Medizinprodukteaufbereitung diskutiert. Die zentrale Erkenntnis der Arbeit ist, dass Smartglasses durchaus im Bereich der AEMP eingesetzt werden können, dafür jedoch technisch noch weiterentwickelt werden müssen. Die in einem Prototypen dargestellten Möglichkeiten für multimediale Anwendungen bereiten der AEMP und insbesondere der medizinischen Medizinprodukteaufbereitung völlig neue Möglichkeiten des freihändigen Arbeitens. In einer Nutzerbefragung der AEMP des Uniklinikums Aachen und einer Befragung technisch versierter  Softwareentwicklerinnen und Softwareentwickler eines Fachunternehmens für Software in der Sterilgutversorgung, wurde der Prototyp evaluiert.

Die Nutzerbefragung ergab, dass der entwickelte Prototyp durchaus sinnvolle multimediale Anwendungen ermöglicht. So können die beiden Hauptfunktionalitäten der Anwendung, also Slideshows und Videofunktionalität in der Praxis sehr gut eingesetzt werden. Insbesondere die Videofunktionalität wurde von den Anwendern als sehr sinnvoll erachtet. 

Die Nutzer hatten jedoch teils starke Probleme mit der eingesetzten Smartglass (Realwear hmt-1). Diese wurde z.T. als zu schwer für einen 8-stündigen Arbeitstag angesehen und manche Probanden bekamen durch den Einsatz des nichttransparenten Displays auf einem Auge Kopfschmerzen. Der grundsätzliche Einsatz von Smartglasses in der AEMP wurde jedoch trotzdem größtenteils positiv beurteilt.
%
%\insertMore{Fazit}