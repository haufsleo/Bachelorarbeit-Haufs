%
%
%
% - - - - - Fazit - - - - - - - - 
%
%
%
\chapter{Fazit}
\label{ch:Fazit}

% 3 Seiten
Im Rahmen dieser Arbeit wurden Smartglasses und insbesondere die Möglichkeiten von Assisted Reality Smartglasses im Bereich der Medizinprodukteaufbereitung diskutiert. Die zentrale Erkenntnis der Arbeit ist, dass Smartglasses durchaus im Bereich der ZSVA eingesetzt werden können, jedoch dafür technisch noch weiterentwickelt werden müssen. Die in einem Prototypen dargestelleten Möglichkeiten für multimediale Anwendungen bereiten der ZSVA und insbesondere der medizinischen Instrumenteaufbereitung völlig neue Möglichkeiten des freihändigen Arbeitens. In einer Nutzerbefragung der ZSVA des Uniklinikums Aachen und einer Befragung technisch versierter  Softwareentwicklerinnen und Softwareentwickler eines Fachunternehmens wurde der Prototyp evaluiert.

Die Nutzerbefragung ergab, dass der entwickelte Prototyp durchaus sinnvolle multimediale Anwendungen ermöglicht. So können die beiden Hauptfunktionalitäten der Anwendung, also Slideshows und Videofunktionalität in der Praxis sehr gut eingesetzt werden. Insbesondere die Videofunktionalität wurde von den Anwendern als sehr sinnvoll erachtet. 

Nutzer hatten jedoch teils starke Probleme mit der eingesetzten Smartglass (Realwear hmt-1). Diese wurde z.T. als zu schwer für einen 8-Stündigen Arbeitstag angesehen und manche Probanden bekamen durch den Einsatz des nichttransparenten Displays auf einem Auge Kopfschmerzen. Der grundsätzliche Einsatz von Smartglasses in der ZSVA wurde jedoch trotzdem größtenteils positiv beurteilt.
%
\insertMore{Fazit}