%
%
%
% - - - - - Evaluation des Prototyps - - - - - - - - 
%
%
%
\chapter{Evaluation des Prototyps}
\label{ch:Evaluation_des_Prototyps}
Zur Evaluation des in Kapitel \ref{ch:Prototyp} beschriebenen Prototyps wurde eine Befragung von potentiellen Nutzern, einem Experten im Bereich der ZSVA sowie einer Gruppe technisch versierter Personen durchgeführt. Befragt wurden zwei
Mitarbeitende der ZSVA, 
der verantwortliche Leiter der ZSVA des Uniklinikums Aachen, 
sowie zwei Softwareentwicklerinnen und Softwareentwickler eines spezialisierten Unternehmens (\emph{iT4Process}), welches Software für Kliniken und insbesondere für die ZSVA von Kliniken entwickelt, der Chief Information Officer (CIO) des Unternehmens sowie ein englischsprachiger Mitarbeiter mit geringen Deutschkenntnissen.
%
%
%
%
% - - - - - Befragung im Beruflichen Umfeld - - - - - - - - 
%
%
%
\section{Befragung im beruflichen Umfeld}
\label{sec:Befragung_im_Beruflichen_Umfeld}
Zur Evaluation wurde ein Ablaufplan zum Testen des Prototypen sowie ein anschließender Fragenkatalog entwickelt, der auf die zu befragende Person angepasst war. 

Vor dem Interview wurde die betreffende Person gefragt, welche Erfahrung sie bereits mit Smartphone bzw. Smartglass Apps ist. Zu bewerten waren diese beiden Punkte im Bewertungsbereich von 1 bis 10, wobei 10 die höchste Erfahrungsstufe ist.

Der von den Testpersonen durchzuführende Ablauf war stets der gleiche und war in zwei Teile aufgeteilt. Nach einer Einführung der (nicht völlig selbsterklärenden) Funktionen der Brille wurden der Testperson zwei Aufgaben gestellt. Es sollten Slideshow- und Videofunktionalität getestet werden. 

Zunächst sollte der Menüpunkt \enquote{Bilder} aufgerufen werden. In dieser Unteransicht sollte zunächst ein QR-Code eingelesen werden, um eine extern gelagerte Slideshow zu laden. Anschließend sollte eine Bildergallerie geöffnet werden und in dieser vor- \& zurück navigiert werden.

Das Testen der Videofunktionalität war deutlich umfangreicher. Nachdem aus der Hauptansicht der Menüpunkt \enquote{Video} ausgewählt wurde, sollte ein Video aufgezeichnet werden. Anschließend sollte das Video gestartet, pausiert sowie vor- \& zurückgespult werden. Kapitelmarken sollten hinzugefügt und passend benannt werden. Durch Öffnen der Kapitelansicht sollte eines der Kapitel gelöscht und ein anderes ausgewählt werden. Anschließend sollte ein QR-Code eingelesen werden und das darin gekapselte Video mit seinen Kapitelmarken geöffnet werden. Hier sollten ebenfalls die zuvor erwähnten Funktionen durchgeführt werden. Anschließend sollte die im Speicher des Gerätes abgelegte Aufnahme erneut nachgeladen werden.

Die Bearbeitungszeit der Nutzer wurde hierbei festgehalten. Anschließend wurden Fragen zur Usability der Brille, zum User-Interface und der Funktionalität des Prototypen sowie zur Einschätzung der Technologie im allgemeinen gestellt.

Die Anwenderinnen und Anwender wurden gefragt, wie sie die Sprachsteuerung, den Tragekomfort sowie die Einsetzbarkeit der Brille in der ZSVA einschätzen. Wünsche zur Usability der Brille wurden abgefragt. Zum User Interface des Prototypen wurde abgefragt, ob das Design ansprechend und selbsterklärend ist, die App übersichtlich und die Steuerelemente groß genug sind. Die Funktionalität des Prototypen wurde mit der Frage nach der praktischen Einsetzbarkeit der Funktionalitäten der App (Slides/ Video) sowie nach einer fehlerfreien Funktionalität der App abgefragt. Um eine professionelle Einschätzung der Technologie einzuholen wurde gefragt, ob die Technologie der Smartglasses als Zukunftstechnologie angenommen wird, ob der Einsatz in der ZSVA empfehlenswert ist und wo sie in der ZSVA eingesetzt werden können. Ebenfalls wurden die Einschätzung der technischen Beschränkungen von Smartglasses in Bezug auf den Einsatz im Bereich der ZSVA, als auch allgemeiner Natur, erfragt. Abschließend wurde die Einsetzbarkeit von Augmented- und Assisted Reality gegenüber gestellt und diskutiert, welche Technologie besser einsetzbar wäre.
%
%\insertMore{Befragung im Beruflichen Umfeld}
%
%
%
% - - - - - Auswertung der Ergebnisse - - - - - - - - 
%
%
%
\subsection{Beschreibung der Interviews}
\label{sec:Beschreinung_der_Interviews}
Im Folgenden werden die durchgeführten Interviews zusammengefasst. Anschließend werden die Interviews ausgewertet und die entscheidenden Erkenntnisse herausgearbeitet.
%
% - - - - - Leiter der ZSVA - - - - - - - - 
%
\subsubsection{Leiter der ZSVA}
%
Der ZSVA-Leiter beschrieb seine Erfahrung mit Smartphones mit dem Punktwert 5, die Erfahrung mit Smartglasses mit 2.

Die Sprachsteuerung der Brille beschrieb er als angenehm, besonders die schnelle Reaktionszeit wurde positiv bewertet. Insgesamt wurde die Brille jedoch als zu schwer wahrgenommen, ein 8 stündiger Einsatz nicht als realistisch eingeschätzt. Dies wurde jedoch lediglich auf dieses Modell (hmt-1) bezogen, andere Smartglasses mit geringerem Gewicht wurden aus Erfahrung des ZSVA-Leiters als geeigneter eingeschätzt.

Die Brille könne durchaus im unreinen und reinen Bereich eingesetzt werden. Kritisch betrachtet wurde die Problematik des Einsatzes mit einer Gleitsichtbrille, da diese im Test zu Problemen des scharfen Sehens führte.

Gestensteuerung wurde als nicht sinnvoll eingeschätzt, reine Sprachsteuerung jedoch sehr begrüßt. Es wurde besonders die Möglichkeit, freihändig arbeiten zu können hervorgehoben. Externe Peripheriegeräte wie ein externer Barcodescanner oder ein Monitor wurden als positiv bewertet.

Das Design des Prototypen wurde als überaus gelungen bezeichnet, die App sei einfach und schlicht gehalten sowie einfach zu bedienen. Die Idee das Design, an die bereits bestehende Software der ZSVA anzupassen und so eine durchgängige Benutzererfahrung zu gewährleisten, wurde als Anregung einer potentiellen Weiterentwicklung des Prototypen gegeben. Trotz der als intuitiv beschriebenen Benutzeroberfläche wurde angemerkt, dass wohl eine gewisse Einarbeitungszeit vonnöten sei. Angeregt wurde, die Texte nach einer Einarbeitungszeit völlig auszublenden und auf die Erfahrung der Mitarbeiter zu setzen, welche die Bedienung dann beherschten. 

Der Prototyp sei durchaus in seiner Funktionalität einsetzbar. Die Sprachbefehle sollten jedoch einheitlicher gestaltet werden. Eine Kombination von Slideshows und Videos sei denkbar und wünschenswert. Hervorgehoben wurde zudem die nötige Akzeptanz durch die Mitarbeiter und die Bedeutung eines ansprechenden Designs der App, um diese Akzeptanz zu fördern.

Der Prototyp habe fehlerfrei funktioniert. Es sei zu keinen Problemen in der Bedienung gekommen.

Die Frage, ob es sich bei dieser Technologie um eine Zukunftstechnologie handle wurde klar mit ja beantwortet. Der Nutzen der Anwendung sei klar zu erkennen, die Auswahl der Brille jedoch entscheidend. Es solle eine leichte Brille gewählt werden, welche zudem mindestens einen 8 Stunden andauernden Arbeitstag ohne Akkuprobleme überstehen könne. Welche Brille eingesetzt werde, hinge völlig von dem Anwendungszweck ab. So sei ebenfalls der Einsatz \emph{echter} AR-Brillen denkbar, jedoch müsse die entsprechende Brille dann auch die äußeren Anforderungen erfüllen. Eine Sprachsteuerung der Brille und somit das beidhändige Arbeiten wurde als absolut notwendig beschrieben. 

Der Einsatz einer Smartglass mit den im Prototypen dargestellten Funktionen sei sowohl im Reinen- als auch im Unreinen Bereich sinnvoll. Als weitere Einsatzgebiete wurden der Zentral-OP sowie Schulungen der ZSVA herausgestellt. Ärzte seien zum Teil nicht mit modernen Instrumenten und deren Verwendung vertraut. Eine Smartglass könne hier helfen. Die Kommunikation zwischen Außendienstmitarbeitern und einem Arzt könne ebenfalls durch Smartglasses verbessert werden. Angeregt wurde zudem die Remotesteuerung der Brille durch Vorgesetzte.

Grenzen des Einsatzes seien für ihn nicht absehbar, da ihm der nötige Erfahrungsschatz fehle.

Ein Vergleich mit der ihm bekannten Microsoft Hololens ergab, dass durch diese intensive Schulungen und vielfältige Einsatzmöglichkeiten in der ZSVA denkbar seien. Der Preis spiele dabei keinerlei Rolle. Problematisch sei bei der Hololens jedoch die mangelnde Sterilisierbarkeit.

Eine Smartglass müsse die strengen Validierungsprozesse der ZSVA durchlaufen und neue Standards dafür definiert werden. So könne er sich vorstellen, dass eine reine Wischdesinfektion der Brille bei entsprechender Dokumentation und Definition in den Validierungsstandards des Uniklinikums ausreichen könne.

Je nach Anwendungszweck müsse also eine andere Smartglass eingesetzt werden. So habe jede Brillenart (Assisted Reality und Augmented Reality) seine Vorteile. Eine Assisted Reality Smartglass mit ihrem relativ kleinen Bildschirm könne ebenso eingesetzt werden wie eine AR-Brille. Dabei sei eine Assisted Reality Brille durch die Sterilisierbarkeit und das vergleichsweise geringe Gewicht zu bevorzugen, wie auch die geringere Ablenkung durch ein transparentes bzw. kleineres Display.
%
% - - - - - Mitarbeiterin der ZSVA - - - - - - - - 
%
\subsubsection{Mitarbeiterin der ZSVA}
%
Eine relativ junge Mitarbeiterin (etwa 30 Jahre) beschrieb ihre Erfahrung mit Smartphones als durchschnittlich (etwa 6), Smartglasserfahrung war nicht vorhanden (1).

Die Sprachsteuerung der Brille sei leicht und einfach, der Tragekomfort jedoch sehr unangenehm und drückend. Zudem habe sie Kopfschmerzen und Konzentrationsprobleme gehabt.

Die Einsetzbarkeit innerhalb der ZSVA beschrieb sie als kritisch und pessimistisch, da fast alle Funktionen der Brille auch an einem stationären Computer durchgeführt werden könnten. 

Beim Einsatz einer Smartglass sei eine Steuerung per Gesten wünschenswert, Toucheingaben wurden zudem als sinnvoll, ein externer Scanner jedoch als überflüssig angesehen. Externe Monitore könne sie sich sehr gut vorstellen. Die Datenbrille habe ein zu kleines Display und die Ränder der Brille seien, wenn dies auch an der \enquote{schweren} Einstellbarkeit geschuldet sein könne, unscharf und dadurch Texte der Ränder schwer lesbar.

Die praktische Einsetzbarkeit der Brille sah sie als kritisch an. So könne sie sich den Einsatz einer Brille, welche ein Display an nur einem Auge habe, nur schwer vorstellen, da dies zu einer Überforderung der Mitarbeiter führte und sehr anstrengend sei. Des Weiteren wies sie auf die mögliche und potentiell gefährliche Strahlung der WLAN- und Bluetoothmodule hin.

Der Prototyp habe fehlerfrei funktioniert. Besonders positiv sei die Videofunktion der Brille, da es so möglich sei, ähnlich einer ihr bekannten Kamera namens \emph{Gopro}, Videos im Blickfeld des Nutzers aufzuzeichnen. 

Ob es sich bei der im Prototypen beschriebenen Technologie jedoch um eine Zukunftstechnologie handle, sei für sie schwer einzuschätzen. Sie stünde dem beschriebenen Prototyp sehr kritisch gegenüber und sehe, dass eher die Nachteile einer solchen Technologie überwögen. Zudem sei die gefährliche Strahlung nicht zu unterschätzen. Die im Prototyp beschriebene Technologie und Funktionalität sei grundsätzlich sinnvoll. Der Einsatz im reinen wie unreinen Bereich der ZSVA sei gleichermaßen kritisch zu beurteilen, da eine Ablenkung durch die Smartglass zu schwerwiegenden Fehler führen könne.

Der Einsatz einer Offline-Version der Brille wertete sie positiv, da so die Strahlungsangst überwunden werden könne. Ob der Einsatz einer \emph{echten} AR-Brille sinnvoller sei könne sie nicht beantworten.
%
% - - - - - Mitarbeiter der ZSVA - - - - - - - - 
%
\subsubsection{Mitarbeiter der ZSVA}
%
Ein etwa 50 Jahre alter Mitarbeiter mit großer Berufserfahrung beschrieb seine Kenntnisse und Erfahrungen mit Smartphones als überdurchschnittlich (etwa 7) sowie die Erfahrung mit Smartglasses als nicht vorhanden (1).

Die Sprachsteuerung der Brille wurde als gewöhnungsbefürftig, jedoch als durchaus erlernbar beschrieben. Die Sprachsteuerung funktioniere nicht immer wie gewünscht auf anhieb, sei jedoch durchaus zufriedenstellend. Auf den Tragekomfort der Brille sei nicht geachtet worden und daher nicht beantwortbar. Den Einsatz der Brille in der ZSVA könne er nicht beurteilen, jedoch sei grundsätzlich solch eine Technologie in der Theorie vorstellbar. Positiv sei, dass so ohne Einsatz der Hände ein Computer bedient werden könne. So sei besonders im Reinbereich der ZSVA ein Beidhändiger Einsatz zwingend nötig, was bei den bisherigen Technologien nicht möglich sei. 

Externe Peripheriegeräte seien durchaus denkbar, da besonders ein externer Monitor die Funktionen der Brille deutlich erweitern könne. Unsinnig sei jedoch der Einsatz eines externen Barcode oder QR-Code-Readers.

Das Design des Prototypen wurde als überaus ansprechend jedoch für unerfahrene Nutzer als gewöhnungsbedürftig bezeichnet. Ob das User-Interface selbsterklärend sei könne er nicht beurteilen. Die Brille sei nicht ideal eingestellt gewesen. Dadurch seien die Ränder der Anwendung nicht sichtbar gewesen. Bei korrekter Einstellung der Brille sei das User-Interface jedoch sehr gut bedienbar. Um die Anwendung gut einsetzen zu können sei sicherlich eine Einarbeitungszeit nötig. 

Der Prototyp sei sicherlich praktisch einsetzbar und biete eine deutliche Erleichterung im Arbeitsalltag der Mitarbeiter. Es sei jedoch eine kurze Anleitung und ausführlichere Erklärung der Anwendung nötig. Nach längerer, alltäglicher Nutzung, ginge die Bedienung jedoch sicherlich \enquote{ins Blut über} und sei dann sicherlich auch einfacher zu bedienen. 

Als negativ wurde der kleine Bildschirm der Brille angemerkt. 

Ob es sich um eine Zukunftstechnologie handle wurde klar mit ja beantwortet. Der Prototyp habe \enquote{Hand und Fuß}. Kritisch angemerkt wurde jedoch, dass bei einem Ausfall der Netzwerkverbindung die Arbeit in der ZSVA nicht mehr möglich sei und so eine starke Abhängigkeit von der Technik entstünde.

Der Einsatz in der ZSVA wurde sehr positiv beurteilt und sowohl im reinen wie auch im unreinen Bereich ein Einsatz als möglich gesehen.
%
% - - - - - Softwareentwicklerin iT4Process - - - - - - - - 
%
\subsubsection{Softwareentwicklerin von iT4Process}
%
Eine Mitarbeiterin von iT4Process, einem Unternehmen welches auf Software für Kliniksoftware und insbesondere für die ZSVA spezialisiert ist, schätzt ihre Erfahrung mit Smartphones auf sehr erfahren ein (9-10), ihre Erfahrung mit Smartglasses schätzt sie als mittelmäßig (etwa 3-4) ein. 

Die Benutzung der Brille wurde als angenehm und einfach zu steuern empfunden. Die Sprachsteuerung hat fehlerfrei funktioniert. Kritisch angemerkt wurde, dass bei zu ähnlichen Worten möglicherweise das falsche Wort gewählt wird. Der Tragekomfort der Brille wurde als ausreichend beschrieben, was jedoch bei einem 8 Stündigen Einsatz möglicherweise kritischer sein könnte. Der Tragekomfort wurde mit dem einer schweren Basecap verglichen. Bei der Nutzung einer Gleit- oder Lesebrille könne die Brille evtl. zu Problemen führen. Bei gesunden Augen sei dies jedoch kein Problem. Sie schätze es so ein, dass eher junge Mitarbeiter der ZSVA die Brille nutzen sollten. Die Nutzung mit einem Auge sei gewöhnungsbedürftig und könne möglicherweise zu Kopfschmerzen führen. Begrüßt wurde hier die Funktionalität der Brille, dass der Bügel der Brille gewechselt werden könne und so zwischen den Augen gewechselt werden kann. Es wurde angemerkt, dass möglicherweise ein transparentes Display mögliche Kopfschmerzen und Ermüdungserscheinungen des Auges ebenfalls verringern könnten.

Ermüdung der Kopfschmerzen wurden nicht festgestellt. Die Sorge vor einer \enquote{gefährlichen Strahlung} bestünde ihrer Meinung nach nicht.

Der Einsatz in der ZSVA wurde als potentiell möglich eingeschätzt. Hervorgehoben wurde die Tatsache, dass die Hände bei der Benutzung frei seien. Die Sprachsteuerung sei dabei besonders praktisch und könne möglicherweise durch Kopfgesten ergänzt werden. Eine Sprachsteuerung solle über eine individuelle Sprachsteuerung, die auf die Stimme des Nutzers personalisiert sei, ergänzt werden. Dadurch könnten Fehler vermieden werden.

Einsatz von Peripheriegeräten wie Monitor oder Bar- und QR-Codescanner wurden begrüßt.

Der Prototyp sei selbsterklärend und erinnere an Apps bekannter mobiler  Betriebssysteme wie Android oder iOS. Die Slideshow lasse sich so leicht bedienen wie eine Powerpoint-Präsentation. Positiv hervorgehoben wurde die Übersichtlichkeit der Bedienelemente. So sei die Anordnung intuitiv und die Wahl einer Links-Rechts-Trennung sehr ansprechend. Die Anzeige vieler Steuerelemente sei besser, als viele Untermenüs. Dabei komme es natürlich darauf an, wie geübt ein Nutzer mit Software sei. Die Elemente seien groß genug. 

Der Prototyp wurde als überaus sinnvoll einsetzbar beschrieben. Der Einsatz des Prototypen bringe eine Fehlerreduktion mit sich und vermöge Unsicherheiten im Alltag der Mitarbeiter zu reduzieren. Zudem sei der Einsatz des Prototypen eventuell für schüchterne Mitarbeiter besonders gut geeignet, da so Unsicherheiten ohne Kommunikation mit Kollegen geklärt werden könnten.

Der Prototyp habe fehlerfrei funktioniert und wurde grundsätzlich sehr positiv aufgenommen. Die klare Trennung von Slideshow- und Videofunktionalität wurde begrüßt, wobei angeregt wurde, die Anwendung um die Möglichkeit von Screenshots der Videos zu ergänzen. So könnten Slideshows mit Videos verbunden werden. Ebenso wurde angeregt, die Möglichkeit zum Versenden von Videos zu ergänzen.

Die Frage, ob der Prototyp eine Zukunftstechnologie beschreibe wurde klar mit ja beantwortet. Die Möglichkeiten des Einsatzes seien sehr vielfältig. Im beruflichen Kontext seien auch die im privaten Umfeld hohen Datenschutzrechtlichen Bedenken nicht so relevant, da eine vorherige Einwilligung der betroffenen Personen eingeholt werden könne. Der Aspekt der freien Hände sei besonders hervorzuheben. Der Prototyp sei bereits jetzt besonders ausgereift und für einen Prototpen sehr funktionell ausgestattet.

Die technischen Grenzen wurden eher auf der Seite der Brille als auf Anwenderseite gesehen. So sei die Sprachsteuerung leicht durch externe Befehle manipullierbar und würde leicht irritieren. Hier sei seitens der Brillenhersteller dringend Nachbesserungsbedarf. 

Der Unterschied zwischen \emph{echter} AR-Brillen und Assistent Reality Brillen war bekannt, ebenso die Microsoft Hololens. Die Hololens wurde als kritisch beurteilt, da der Einsatz dieser AR-Brille eine zu große Ablenkung für die Mitarbeiter dar und außerdem sei sie zu schwer. Ebenso sei der Effekt der Virtual Sickness nicht zu ignorieren. Der Vorteil von Assistent Reality Brillen sei die leichte Bedienbarkeit für die Nutzer und das geringere Gewicht der Brille.
%
% - - - - - Fremdsprachiger Softwareentwickler iT4Process - - - - - - - - 
%
\subsubsection{Fremdsprachiger Softwareentwickler von iT4Process}
%
Ein englischsprachiger Mitarbeiter von iT4Process, welcher als nicht-Muttersprachler nur geringe Deutschkenntnisse, wurde ebenfalls befragt. Laut seiner Selbsteinschätzung besitzt er fortgeschrittene Kenntnisse im Umgang mit Smartphones (8). Seine Erfahrung mit Smartglasses sind dagegen sehr gering (2). Dieser Nutzer wurde ausgewählt, da viele Mitarbeiter der ZSVA nur über geringe Deutschkenntnisse verfügen und daher mit starkem ausländischen Akzent reden.

Im Bereich der Spracherkennung tauchten Probleme auf. Aufgrund der mit starkem Akzent versehenen Aussprache kam es immer wieder zu Verständigungsproblemen. So musste der Proband Sprachbefehle mehrfach wiederholen. Mit einiger Übung wurden diese Probleme jedoch weniger. Zudem musste bei erster Anwendung viele der Funktionen der App erklärt werden, da viele der Befehle nicht sofort verstanden wurden.

Der Tragekomfort sowie das Gewicht wurden als angenehm empfunden. Es bestand jedoch ein Problem mit der vom Probanden getragenen Brille. So konnte aufgrund der eigenen getragenen Brille das Bild der Smartglass nicht scharf fokussiert werden.

Das Design des Prototypen wurde als sauber und selbsterklärend bezeichnet. Angeregt wurde, dass besser Buttons mit Rand benutzt werden sollten, da nicht genau klar sei, was Buttons und was Text ist. Steuerelemente seien groß genug sowie übersichtlich angeordnet. Übermüdung oder Kopfschmerzen traten nicht auf. Der Prototyp wurde als gut benutzbar beschrieben und sei mit etwas Hilfe auch gut verständlich. Er sei nach der Einweisung auch selbstständig nutzbar.

Problematisch war der hmt-1-Befehl \enquote{Hauptmenü}, welcher die App vollständig beendet, sodass der lokale Server neu initialisiert werden müsste.
%
% - - - - - Softwareentwickler von iT4Process - - - - - - - - 
%
\subsubsection{Softwareentwickler von iT4Process}
Ein Softwareentwickler der Firma iT4Process schätzte seine Smartphoneerfahreungen als sehr gut ein (8) sowie seine Smartglasserfahrungen als sehr gering bzw. nicht existent ein (1).

Zunächst stellte er fehst, dass bei nicht optimaler Einstellung der Brille die Ränder nicht sichtbar sind. Die Sprachsteuerung wurde als teilweise schwierig bis nicht funktionierend beschrieben. Sprach er langsam und ordentlich war die Spracherkennung erfolgreich. Die Brille wurde als bequem beschrieben, der 8 Stündige Dauereinsatz jedoch in Frage gestellt. Die Tatsache, dass ein Auge permanent mit einem Display bedeckt ist, wurde stark kritisiert. Ein transparentes Display könne hier Abhilfe schaffen.

Der Einsatz von Smartglasses in der ZSVA wurde als grundsätzlich sinnvoll erachtet, die mögliche Störung und Ablenkung der Mitarbeiter jedoch sehr kritisch gesehen. Die 3D-Sicht des Auges sei stark eingeschränkt und behindere so bei filigranen Tätigkeiten, da die Präzision eingeschränkt sei.

Gestensteuerung sowie Kopfbewegungssteuerung wurden als sinnvoll angesehen, wohingegen Touch-Steuerung abgelehnt wurde. Als Peripheriegeräte wurden externe Scanner sowie Monitore für sinnvoll befunden.

Das Design des Prototypen wurde als selbsterklärend beschrieben. Es sei jedoch nicht immer klar, welche Funktionen wie erreichbar seien, was zu einer Überforderung führe. Die Verteilung der Steuerelemente sei nicht immer perfekt und daher etwas verwirrend. Die Elemente seien jedoch gut lesbar.

Als praktisch einsetzbar wurden die Aufnahme und das Abspielen von Videos sowie das Setzen von Kapitelmarken bezeichnet. Statt Slideshows wurden Videos als deutlich sinnvoller angesehen. Die Funktionalität des verlangsamten Abspielens habe gefehlt. Positiv wurde die Strukturierung sowie die Ebenenanzahl beschrieben. Man wisse stehts wozu welche Funktionalität geeignet sei.

Smartglasses wurden als klare Zukunftstechnologie bezeichnet, die verwendete hmt-1 jedoch nicht, da die Spracherkennung nicht genügend und die Displayschärfe nicht ausrechend sei. Der Prototyp beschreibe ebenfalls eine Zukunftstechnologie. Die Funktionen des Prototypen wurden für den Einsatz bei Anleitungen, beim Anlernen neuer Mitarbeiter sowie für Schritt für Schrittanleitungen gesehen. Die Anwendung könne sowohl im Reinen als auch im Unreinen Bereich Verwendung finden. Andere Anwendungsbereiche seien Gebäudeführung sowie alle Arbeiten, bei denen Angestellte nachträglich geschult werden müssten. 

Der Technologie seien aber mannigfaltige Grenzen gesetzt: So sei die Bandbreite des WLAN, die Positionsbestimmung ohne GPS (in Gebäuden), die Größe der Instrumente sowie der Akku limitierende Faktoren. Der Akku müsse zudem mindestens 8 Stunden halten, was nach mehrjähriger Nutzung problematisch werde. So müsse eine Aufladeinfrastruktur geschaffen werden.

Der Vergleich von AR und Assisted Reality zeige, dass es stark auf den Anwendungsfall ankomme. So benötigten AR Brillen deutlich mehr Strom (gesteigerter Akkuverbrauch) sowie mehr Rechenleistung. Asssted Reality Brillen hätten den Vorteil geringerer Rechenleistung bei verringerter Ablenkung des Nutzers zu haben.
%
% - - - - - Chief Information Officer iT4Process - - - - - - - - 
%
\subsubsection{Chief Information Officer iT4Process}
Der Chief Information Officer der Firma iT4Process beschreibt sich selbst als sehr erfahren mit Smartphones (8) und mittel erfahren mit Smartglasses (4).

Die Sprachsteuerung der hmt-1 wurde als \enquote{erstaunlich gut} bezeichnet. Sie übertraf die Erwartungen des Probanden. Der Tragekomfort der Brille wurde positiv aufgenommen. So wurde die Brille als gut ausbalanciert bezeichnet. Das Display wurde kritisch betrachtet. So wäre eine Abwägung zwischen einen halbtransparenten und transparenten Display zu treffen. 

Der Einsatz in der ZSVA wurde differenziert betrachtet: die Tatsache, dass Nutzer sich ins Gesicht fassen müssen, um die Brille zu justieren wurde kritisiert und dass im Reinbereich die Brille nicht wirklich gut einsetzbar sei. Spracheingabe wurde als ermüdend und störend eingestuft, was eine Toucheingabe (möglicherweise mittels Peripheriegeräten) sinnvoll mache. Spracherkennung könne sich gegenseitig behindern, da mehrere Menschen parallel arbeiteten. So könne beispielsweise ein Fußschalter helfen oder andere Peripheriegeräte zur Eingabe (bspw. Touchbar). Externe Scanner wurden genau wie externe Monitore als sinnvoll angesehen. Eine Waage könne ebenfalls eine sinnvolle Erweiterung darstellen.

Das Design des Prototypen wurde mit iOS-Apps verglichen, da es von typischen Android Apps abweicht. Die Farbgebung wurde daher für Androidnutzer als irritierend wahrgenommen, da die Farbgebung anders als typische Android-Apps ausfällt. Zur genaueren Identifikation der Kapitelmarken wurden Screenshots empfohlen, welche neben der Auswahl in der Kapitelauswahl angebracht werden sollten. Zur Steuerung der App wurde eine Kopfgestensteuerung angeregt, die ein vor- und zurückspulen ermöglichen könne. Auch hier kam es zur falschen Benutzung des hmt-1-Befehls \enquote{Hauptmenü}, was zu einem Beenden der App führte. Die Steuerelemente wurden als groß genug bezeichnet.

Die praktische Einsetzbarkeit wurde vor allem bei Schulungen gesehen. Die App wurde als fehlerfrei bedienbar beschrieben. Als gut wurde die geringe Menütiefe hervorgehoben.

Ob es sich um eine Zukunftstechnologie handle könne der Proband nicht sagen, es sei noch ein \enquote{weiter Weg} bis zur Marktreife. Der Haupteinsatzort sehe er im Lager und in der Chargenfreigabe der ZSVA. Für genauere Anleitungsvideos seien Monitore besser geeignet. Smartglasses und multimediale Anwendungen seien immer da gut einsetzbar, wo kein fester Arbeitsplatz vorhanden sei. So könne freihändig gearbeitet werden.

Grenzen der Smartglass in der ZSVA seien durch die beschränkten Möglichkeiten der Interaktion ohne Maus und Tastatur sowie die geringe Auflösung der Kamera gegeben. Die Informationsdichte auf einem kleinen Display sei stark limitiert und bechränke damit die Menge an Informationen, die auf der kleinen Fläche dargestellt werden könne. Positiv sei, dass in der ZSVA Aachen beispielsweise eh nur sehr kleine Bildschirme mit geringer Auflösung eingesetzt würden, was eine Übertragung auf eine Smartglass vereinfache.

Vergleiche zwischen Assisted Reality Brillen und Augmented Reality Brillen zeige, dass die Voraussetzungen (Abwischbar, großes Sichtfeld und leichtes Gewicht) momentan echte AR-Brillen als nicht geeignet machen, um in der ZSVA eingesetzt zu werden. Der Vorteil von Assisted Realitybrillen sei das geringe Gewicht, die Abwischbarkeit sowie die Robustheit der Brille.
%
%
% - - - - - Auswertung der Ergebnisse - - - - - - - - 
%
\subsection{Auswertung der Ergebnisse}
\label{sec:Auswertung_der_Ergebnisse}
%
%
%
Die Erfahrung der Testpersonen mit der verwandten Technologie der Smartphones (oftmals mit dem bekannten Betriebssystem Android) kann als durchaus überdurchschnittlich angesehen werden. So geben alle Testpersonen der ZSVA an, über leicht über dem Durchschnitt ($>5$) liegende Erfahrung mit dieser Technologie zu verfügen. Die Erfahrung mit Smartglasses ist jedoch bei fast allen Nutzern nicht vorhanden. Lediglich der Leider der ZSVA gab an, geringe (2) Erfahrung mit Smartglasses zu haben. Die befragten Softwareentwickler verfügten berufsbedingt über einen größeren Erfahrungsschatz.

Die Hardware, also die hmt-1 Brille, wurde durchaus kritisch gesehen. So sei sie zu schwer, verursache Kopfschmerzen und wurde zum Teil als unbequem bezeichnet. Das Display wurde als zu klein, schwer einstellbar und nicht scharf beschrieben. Die Einsetzbarkeit der verwendeten Brille wurde ebenfalls insgesamt als kritisch angesehen. Der Leiter der ZSVA jedoch beschrieb die ihm bekannten anderen Smartglasses (sofern diese sterilisierbar wären) als durchaus geeigneter. So komme es auf das Gewicht der Brillen an. 

Peripheriegeräte seien zum Teil sinnvoll. Monitore wurden insgesamt als gute Erweiterung eingeschätzt. Touch und Gestensteuerung fanden dagegen ambivalente Bewertung der Tester.

Der Prototyp wurde durchgehend positiv angesehen. Besonders der verantwortliche Leiter der ZSVA sah eine große Chance der im Prototypen beschriebenen Technologien. So könnten große Arbeitserleichterungen durch den multimedialen Einsatz in der ZSVA entstehen. 

Das User Interface des Prototypen wurde ebenfalls durchgehend als ansprechend und größtenteils als Selbsterklärend beschrieben. Die Funktionalitäten des Prototypen wurden selbst von der der Technik kritisch gegenüberstehenden Mitarbeiterin als sinnvoll angesehen. Das User Interface des Prototypen wurde durchgehend als übersichtlich und ansprechend beschrieben.

Die Funktionalität der Brille sei gut bedienbar und fehlerfrei.

Die Technologie wurde größtenteils als Zukunftstechnologie bezeichnet. Der Einsatz in der ZSVA wurde teils zwar als kritisch beurteilt, jedoch grundsätzlich als möglich eingestuft. Der Einsatz im Reinbereich und unreinen Bereich der ZSVA sei möglich. Besonders hervorzuheben ist hier die professionelle Einschätzung des Leiters der ZSVA und des erfahrenen Mitarbeiters, welche den Einsatz in beiden Bereichen als überaus sinnvoll angesehen haben.

Andere Einsatzmöglichkeiten innerhalb der ZSVA wurden als durchaus realistisch eingeschätzt. So wurde vor allem der Aspekt der Schulung und Fortbildung genannt. Hier könnten auch \emph{echte} AR-Brillen eingesetzt werden, welche ansonsten für den ganztägigen Einsatz zu schwer seien.

Die Grenzen der Assisted Reality-Brillen wurden durchaus gesehen. So sei die geringe Displaygröße ein starker limitierender Faktor der Einsatzmöglichkeiten. So kann sich der Leiter der ZSVA den Einsatz von \emph{echten} AR-Brillen besser vorstellen, sofern die äußeren limitierenden Faktoren wie Sterilisierbarkiet eingehalten werden können.

Ob eine \emph{echte} AR-Brille oder eine Assisted Reality Brille mit der im Prototypen dargelegten multimedialen Technologie eingesetzt werde, hänge ganz vom Anwendungszweck ab und sei pauschal nicht zu sagen. Die Vorteile von Assisted Reality-Brillen im Bereich der Medizinprodukteaufbereitung wurden jedoch bei allen Testern positiv beurteilt.

Die Nutzererfahrung war sehr davon abhängig von der Erfahrung der Nutzer mit Smartphones und anderen Smartglasses im Allgemeinen war. So waren erfahrene Nutzer deutlich zufriedener mit der Brille und mit dem Prototypen und konnten wesentlich besser die Relevanz der Software einschätzen. 