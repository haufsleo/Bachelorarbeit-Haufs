%
%
%
% - - - - - Evaluation des Prototyps - - - - - - - - 
%
%
%
\chapter{Evaluation des Prototyps}
\label{ch:Evaluation_des_Prototyps}
Zur Evaluation des in Kapitel \ref{ch:Prototyp} beschriebenen Prototyps wurde eine Befragung von potentiellen Nutzern, einem Experten im Bereich der ZSVA sowie einer Gruppe technisch versierter Personen eruiert und evaluiert. Befragt wurden
Mitarbeitende der ZSVA, 
der Verantwortliche Leiter der ZSVA des Uniklinikums Aachen, 
sowie drei Softwareentwicklerinnen und Softwareentwickler eines spezialisierten Unternehmens (\emph{iT4Process}), welches Software für Kliniken entwickelt.
%
\insertMore{Evaluation des Prototyps}
%
% 1 Seite
%
%
%
% - - - - - Befragung im Beruflichen Umfeld - - - - - - - - 
%
%
%
\section{Befragung im Beruflichen Umfeld}
\label{sec:Befragung_im_Beruflichen_Umfeld}
Zur Evaluation wurde ein Ablaufplan zum Testen des Prototypen sowie ein anschließender Fragenkatalog entwickelt, welcher auf die zu befragende Person angepasst war. 

Vor dem Interview wurde die betreffende Person gefragt, wie sie ihre Erfahrung mit Computertechnologie grundsätzlich einschätzt und wie ihre Erfahrung mit Smartphone bzw. Smartglass Apps ist. Zu bewerten waren diese drei Bereiche im Bewertungsbereich von 1 bis 6, wobei 6 die höchste Erfahrungsstufe ist.

Der von den Testpersonen durchzuführende Ablauf war stets der gleiche und war in zwei Teile geteilt. Nach einer Einführung der (nicht völlig selbsterklärenden) Funktionen der Brille wurden der Testperson zwei Aufgaben gestellt. Es sollten Slideshow- und Videofunktionalität getestet werden. 

Zunächst sollte der Menüpunkt \enquote{Bilder} aufgerufen werden. In dieser Unteransicht sollte zunächst ein QR-Code eingelesen werden, um eine extern gelagerte Slideshow zu laden. Anschließend sollte eine Bildergallerie geöffnet werden und in dieser Vor- \& zurück navigiert werden.

Das Testen der Videofunktionalität war deutlich ausführlicher. Nachdem aus der Hauptansicht der Menüpunkt \enquote{Video} ausgewählt wurde, sollte ein Video aufgezeichnet werden. Anschließend sollte das Video gestartet, pausiert sowie vor- \& zurückgespult werden. Kapitelmarken sollten hinzugefügt und passend benannt werden. Durch Öffnen der Kapitelansicht sollte eines der Kapitel gelöscht werden und ein anderes Kapitel ausgewählt werden. Anschließend sollte ein QR-Code eingelesen werden und das darin gekapselte Video mit seinen Kapitelmarken geöffnet werden. Hier sollten ebenfalls die zuvor erwähnten Funktionen durchgeführt werden. Anschließend sollte die im Speicher des Gerätes abgelegte Aufnahme erneut nachgeladen werden.

Die Bearbeitungszeit der Nutzer wurde hierbei festgehalten. Anschließend wurden Fragen zur Usability der Brille, zum User-Interface und der Funktionalität des Prototypen sowie zur Einschätzung der Technologie im Allgemeinen gestellt.

Die Anwenderinnen und Anwender wurden gefragt, wie sie die Sprachsteuerung, den Tragekomfort sowie die Einsetzbarkeit der Brille in der ZSVA einschätzen. Wünsche zur Usability der Brille wurden abgefragt. Zum User Interface des Prototypen wurde abgefragt, ob das Design ansprechend und Selbsterklärend ist, die App übersichtlich und die Steuerelemente groß genug sind. Die Funktionalität des Prototypen wurde mit der Frage nach der praktischen Einsetzbarkeit der Funktionalitäten der App (Slides/ Video) sowie nach einer Fehlerfreien Funktionalität der App abgefragt. Um eine professionelle Einschätzung der Technologie einzuholen wurde gefragt, ob die Technologie der Smartglasses als Zukunftstechnologie angenommen wird, ob der Einsatz in der ZSVA empfehlenswert ist und wo sie in der ZSVA eingesetzt werden können. Ebenfalls wurden die Einschätzung der technischen Beschränkungen von Smartglasses in Bezug auf den Einsatz im Bereich der ZSVA, als auch allgemeiner Natur. Abschließend wurde die Einsetzbarkeit von Augmented- und Assisted Reality gegenüber gestellt und diskutiert, welche Technologie besser einsetzbar wäre.
%
\insertMore{Befragung im Beruflichen Umfeld}
% 3 Seiten
%
%
%
% - - - - - Auswertung der Ergebnisse - - - - - - - - 
%
%
%
\subsection{Auswertung der Ergebnisse}
\label{sec:Auswertung_der_Ergebnisse}
\insertMore{Auswertung der Ergebnisse}
% 3 Seiten