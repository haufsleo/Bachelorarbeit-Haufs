%
%
%
% - - - - - Konzept - - - - - - - - 
%
%
%
\chapter{Konzept}
\label{ch:Konzept}
% 0,5 Seiten
Im Folgenden wird ein Konzept vorgestellt, wie Smartglasses in der Sterilgutversorgung eingesetzt werden können. Dazu werden zunächst die Interaktionsmöglichkeiten erläutert, dann die Möglichkeiten der Objekterkennung dargelegt und dies mit möglichen Peripheriegeräten konkretisiert. Eine Beschreibung der Medienverwaltung, -präsentation sowie -erfassung und -erstellung werden abschließend beschrieben.
%
%
%
% - - - - - Interaktionsmöglichkeiten - - - - - - - - 
%
%
%
\section{Interaktionsmöglichkeiten}
\label{sec:Interaktionsmoeglichkeiten}
% 2 Seiten
Wie in Kapitel \ref{sec:Stand_der_Technik} bereits beschrieben gibt es bei Smartglasses verschiedene Arten der Interaktion, und damit der Eingabe an der Brille. Je nach Hardware und aufgespielter Software (App) sind verschiedene Interaktionsmöglichkeiten denkbar. 

So kann die Brille in allen in Kapitel \ref{sec:VergleichSmartglasses} beschriebenen Geräten per Sprachsteuerung bedient werden. Dies ist besonders im Arbeitsumfeld der Medizinprodukteaufbereitung wichtig da, wie bereits ausführlich in Kapitel \ref{sec:Ablauf_der_Medizinprodukteaufbereitung} und \ref{sec:Spezifische_Anforderungen_an_Smartglasses} beschrieben, die Hände der Angestellten frei sein müssen und im Dekontaminationsbereich ein Berühren der Brille mit den potenziell kontaminierten Handschuhen nicht gewünscht ist.

Manche Brillen wie die \emph{Glass} von Google besitzen zusätzlich eine Touchbar, die eine Berührungssensitive Steuerung ermöglichen. Denkbar sind hier bei entsprechender Technik auch evtl. Touch-Gestensteuerung wie \enquote{Pinch-to-zoom}.

Durch die Kamera der Smartglass, die auch alle in Kapitel \ref{sec:VergleichSmartglasses} beschriebenen Geräte haben, lassen sich zudem Gesten- und mithilfe des Bewegungssensors auch Kopfbewegungen erkennen. So lassen sich über Gesten der Hände beispielsweise mögliche virtuelle Elemente in der grafischen Oberfläche der Brille auswählen. Swipe- und Scrollgesten wären so ebenfalls denkbar, wie die zuvor erwähnten Pinch-to-zoom Geste.

Per Bar- und QR-Codes können direkte Befehle an die Brille gegeben werden. So ist es möglich Methoden der App auf der Brille aufzurufen oder Hyperlinks auszuwählen. Es wäre beispielsweise denkbar einen QR-Code zu verwenden, um Videos zu einem chirurgischen Instrument oder OP-Sieb anzuzeigen oder zu einer Montage- oder Packanleitung. Ein Barcode kann als Beschriftung eines Siebes oder Sterilgutes verwendet werden und damit automatisch lokalisiert werden. Barcodes können sind in ihrer Anzahl an Ziffern stark begrenzt.
%
\note{Beschränkung von Barcodes- Ziffernzahl}
QR-Codes sind zweidimensionale (Matrix-) Barcodes und können bis zu 4.296 alphanumerische Zeichen speichern \cite{INCORPORATED2018}. So können auch komplexe Informationen \emph{mit einem Blick} erfasst werden.

Die im folgenden Unterkapitel (\ref{sec:Objekterkennung}) näher beschriebene Objekterkennung ermöglicht ebenfalls starke Eingabemöglichkeiten.
%
%
%
% - - - - - Objekterkennung - - - - - - - - 
%
%
%
\section{Objekterkennung}
\label{sec:Objekterkennung}
% 2 Seiten 
Objekterkennung ist in den letzten Jahren massiv weiterentwickelt worden. Firmen wie Apple \cite{Apple2018b}, Microsoft \cite{Girshick2015} oder Facebook \cite{Schroepfer2015} haben große Fortschritte in der Technologie der Objekterkennung gemacht. Diese großen Meilensteine konnten durch die großen Fortschritte in der Künstlichen Intelligenz mit neuronalen Netzwerken \cite{Apple2018b, Schroepfer2015} erreicht werden. Zur modernen Objekterkennung mittels \emph{Machine Learning-Algorithmen} ist es heute möglich in sehr kurzer Zeit Objekte selbst in komplizierten Umgebungen zu erkennen \cite{Schroepfer2015}. 
Die Qualität von Objekterkennung hat dank Neuronaler Netze eine enorme Qualitätssteigerung erfahren.

Auf Android ist für Objekterkennung momentan eine Internetverbindung nötig, da nicht \emph{on Device}, sondern \emph{cloudbasiert} gearbeitet wird. Dies ist deutlich ineffizienter in der Verarbeitungszeit als \emph{on device}-Objekterkennung, erfordert jedoch eine Internetverbindung. Dafür ist die Qualität der Analyse besser, da eine höhere Rechenleistung der verwendeten Neuronalen Netze Verwendung findet.

Mittels Objekterkennung in Smartglasses eröffnen sich völlig neue Möglichkeiten der Interaktion für Nutzer. So könnten in der Sterilgutversorgung spezifische chirurgische Instrumente oder ganze Siebe mit Instrumenten erkannt werden. Es wäre möglich direkte Handlungsanweisungen für Mitarbeiter der Medizinprodukteaufbereitung im Blickfeld anzuzeigen oder Austauschmöglichkeiten mit Mitarbeitern über konkret erkannte Instrumente zu führen. Es wäre per Objekterkennung ebenfalls möglich, die Dokumentation von Arbeitsschritten mittels erkannter Objekte zu führen, ohne einen Bar- oder QR-Code einzulesen.
%\insertMore{Objekterkennung}
%
%
%
% - - - - - Externe Peripherie - - - - - - - 
%
%
%
\subsection{Externe Peripherie}
\label{sec:Externe_Peripherie}
% 1 Seite 
Smartglasses der in Kapitel \ref{sec:VergleichSmartglasses} vorgestellten Modelle sind allesamt Androidgeräte. Andere Betriebssystemhersteller mobiler Plattformen haben bislang keine Smartglassunterstützung veröffentlicht. Bei Smartglass-Androidgeräten ist es ohne Probleme möglich, externe Peripheriegeräte zu verwenden. Diese können per USB, WLAN oder Bluetooth angebunden werden.  Peripheriegeräte einer Smartglass sollten nach Möglichkeit drahtlos, also mittels WLAN oder Bluetooth mit der Brille verbunden sein, da Kabel an der Brille sehr störend sein könnten. 

Es gibt viele denkbare Peripheriegeräte, von denen hier nur einige wenige vorgestellt werden können.

Mittels eines externen Barcode- oder QR-Scanners könnten entsprechende Eingaben auch dann eingelesen werden, wenn der Angestellte die Codes nicht direkt anschaut. Es kann möglicherweise unpraktisch sein, immer direkt die betreffenden Codes anschauen zu müssen. Dies könnte zu gefährlichen Situationen führen, wenn scharfe oder kontaminierte Instrumente nah ans Auge gehalten werden müssten, um eingelesen zu werden.

Mittels externer (steriler) Bedienelemente könnte die Benutzerführung der Brille verbessert werden. So könnte die oft nicht sterile Touchbar der Brille (vgl. Google Glass) ersetzt werden und so auch eine solche Brillen mit dieser guten Funktionalität ausgestattet werden, die keine Touchbar haben. 

Ein externer Monitor könnte das Sichtfeld des Brillendisplays ergänzen, sodass andere Mitarbeiter (z.B. bei Schulungen, Präsentationen etc.) die gleichen Bilder der AR-Brille sehen, die der Anwender sieht. Wenn ein transparenter Monitor eingesetzt wird, kann der AR-Effekt sogar noch verstärkt werden, da so noch weitere AR-Elemente eingeblendet werden können, die in dem relativ kleinen Bildschirm der Brille nicht möglich sind.

Abschließend sei der Einsatz von Codes genannt, die die Position für Virtuelle Elemente im Raum definieren. So könnten an definierten Stellen im Raum virtuelle Elemente im Bildschirm der Brille eingeblendet werden. Virtuelle \enquote{Hologramme} können im Raum angezeigt werden, die an einer wohl definierten Stelle im Raum platziert werden (bspw. an einer Wand oder in einem Sieb mit chirurgischen Instrumenten).
%
%
%
% - - - - - Medienverwaltung - - - - - - - - 
%
%
%
\subsection{Medienverwaltung}
\label{sec:Medienverwaltung}
% 2 Seiten (???)
Werden Videos oder Bilder auf einer Smartglass erzeugt, können diese entweder (lokal) auf der Brille bzw. deren SD-Karte oder (extern) auf einem Server (\emph{Clouddienst, Server}) gespeichert werden. Werden die Bilder lokal gespeichert, so benötigt die betreffende Anwendung eine entsprechende Datenbankverwaltung der Bilder bzw. Videos. Bilder können im Normalfall in der Originalgröße der Kamera gespeichert werden, Videos müssen dagegen evtl. komprimiert werden, da sie ansonsten zu groß sind um auf einen Server geladen zu werden. 

Steuerung der Eingaben lassen sich auf unterschiedliche Weise regeln. So ist es möglich, Links zu Videos oder Bildern in einer Datenbank zu speichern und per Barcode oder QR-Code darauf zuzugreifen, also den Identifikationsschlüssel des entsprechenden Videos in der Datenbank zu verlinken. Der Link könnte auch zu einem extern gespeicherten Medium in einem QR Code gespeichert und direkt verlinkt werden. Diese Datenbank kann im Gerät oder auf einem externen Server gespeichert sein. 
%
\note{Satzbau}

Kurze Texte wie Beschriftungen können direkt in einem QR-Code (evtl. verschlüsselt) gespeichert werden. So spart sich der Anwender die Verbindung zu einem Server und die dort hinterlegte Datenbank, in der der Text gespeichert wäre.

Zur Persistierung von Objekten im Dateisystem gibt es mehrere Möglichkeiten. Daten können in einer Datenbank gespeichert werden. Dies kann auch mittels \emph{Object relationalem Mapping} (ORM) geschehen.

Ein einfacherer Weg ist die Speicherung der Objekte in serialisierter Form im Dateisystem. Dies sollte jedoch nur für einfachere Programme wie einen Prototypen vgl. Kapitel~\ref{ch:Prototyp} geschehen, da die Skalierbarkeit des Programmes sehr eingeschränkt ist.

Videos und Bilder können entweder gestreamt werden oder im Dateisystem gespeichert werden.
%
%
%
% - - - - - - Medienpräsentation - - - - - - - - 
%
%
%
\section{Medienpräsentation}
\label{sec:Medienpraesentation}
% 2 Seiten
Die Art und Weise, wie Daten auf einem in der Smartglass vergleichsweise kleinen Bildschirm dargestellt werden, ist eine große Herausforderung für die Entwicklung von Smartglass-Apps. Mobile Geräte, wie auch Smartglasses, bieten nur wenig Platz, um Informationen zu präsentieren. Medien müssen übersichtlich und gleichzeitig präsent angeordnet werden. Beschriftungen, die die Sprachbefehle erklären, müssen übersichtlich, aber auch nicht zu präsent angeordnet werden. Die Smartglassanwendung muss, wie jede gute Smartphone App auch, selbsterklärend zu benutzen sein, damit die in Kapitel \ref{sec:Spezifische_Anforderungen_an_Smartglasses} beschriebene gute Nutzererfahrung gewährleistet ist. 
Muss ein technikaffiner Nutzer erst in einem längeren Seminar o.Ä. die Nutzung der App \emph{lernen}, da sie sonst nicht zu verstehen ist, so ist das Design der App ungenügend \cite{Hoober2011} \cite[S.~141ff]{Norman2013}.

Designer von Oberflächen haben nicht die Möglichkeit über großen Platz für Bilder, Listen etc. zu verfügen. Das Design muss dabei auf das Wesentliche reduziert werden. Die erste View der App sollte dabei die meisten Informationen enthalten \cite[S.~442]{Tidwell2005}, um die App zu verstehen. So sollte zur Benutzung der App eine kurze Anleitung angezeigt werden, da Android-Smartglasses nicht selbsterklärend bedienbar sind.

Da es keine Designrichtlinien der Hersteller für Smartglass-Apps gibt, können dank der ähnlichen kompakten Bauweise der Apps hier ansatzweise die Designrichtlinien für Smartwatches als Anhaltspunkt dienen. Diese können nur als geringfügige Idee dienen und müssen natürlich für Smartglasses angepasst werden.
Für \emph{watchOS} hat Apple entsprechende \emph{Human Interface Guidelines} veröffentlicht \cite{Apple2018c}. Google hat für seine Smartwatches mit \emph{Android Wear} eine ähnliches veröffentlicht \cite{Google2018}. Aus diesen beiden Richtlinienkatalogen geht unter anderem Folgendes, auf Smartglasses übertragbares, hervor:
\\
Mobile Geräte mit stark beschränkter Bildschirmgröße  müssen Informationen schnell und einfach anzeigen können. Dabei muss darauf geachtet werden, dass die Inhalte in kompakter Form und sehr auf das Wichtigste reduziert dargestellt werden. Die Oberfläche der Anwendung muss übersichtlich gehalten werden. Dabei müssen Texte und Bilder so angeordnet werden, dass Nutzer die Informationen, die sie benötigen, schnell finden. Es muss darauf geachtet werden, dass Schriftgrößen bei dieser reduzierten Bildschirmgröße gut lesbar sind. Es muss vermieden werden, zu viele Informationen gleichzeitig auf dem Bildschirm anzuzeigen. Benutzer sollen die wichtigsten Informationen sofort sehen, damit sie den Bildschirm nicht mit unwichtigen Details überladen. Es ist wichtig, visuelle Gruppierungen zu erstellen, um den Benutzern das Auffinden der Informationen zu erleichtern. Ebenso ist es erforderlich, die volle Größe des relativ kleinen Bildschirms zu nutzen \cite{Apple2018c, Google2018}.

Klassische Scrolllisten (\emph{Scrollviews}), wie sie bei Smartphones und oftmals auch bei Smatwatches üblich sind, lassen sich bei Smartglasses nicht gleichermaßen einsetzen. Es ist also nötig, neue Konzepte für dieses Steuerelement zu entwickeln. Eine Möglichkeit für Listen mit geringer Länge ist die Steuerung der Liste mittels Kopfbewegungen. So kann die Liste nach unten gescrollt werden, wenn der Kopf sich entsprechend bewegt. Dies lässt sich jedoch nur für kurze Listen einsetzen. Eine weitere Möglichkeit ist der Einsatz von Slideshows (\emph{Pagination}), die über einen Vorwärts- und einen Zurückbutton verfügen. Dies ermöglicht die Navigation von langen Listen. Ist der Anfang der Liste erreicht, so wird der Zurückbutton ausgeblendet. Ist das Ende der Liste erreicht, wird entsprechend der Vorwärtsbutton ausgeblendet.
%
%
%
% - - - - - Medienerfassung/-erstellung - - - - - - - - 
%
%
%
\section{Medienerfassung/-erstellung}
\label{sec:Medienerfassung_-erstellung}
% 2 Seiten
Medien in einer Smartglass können auf unterschiedlichste Weise erfasst werden. Zum einen können Videos und Bilder aufgenommen werden. Sprachaufnahmen sind ebenfalls möglich, genau wie in Text transkribierte Sprache.

Medien, die auf einer Smartglass erzeugt werden, müssen ggf. komprimiert werden oder anders weiterverarbeitet werden, bevor sie verwendet werden können. So müssen Video- und Audioaufnahmen ggf. digital aufbereitet werden, da in der Sterilgutversorgung eine sehr starke Geräuschkulisse herrscht. Monotone Hintergrundgeräusche lassen sich heute sehr gut herausfiltern. Hier ist es natürlich hilfreich, wenn die Smartglass von sich aus schon über eine Geräuschunterdrückung (\emph{Noise Cancellation}) verfügt.

Video- und Fotoaufnahmen müssen in hoher Auflösung erfolgen, da bei der Erkennung von QR- und Barcodes hohe eine gewisse Mindestauflösung erforderlich ist. Dies ist nötig, da die OP-Siebe in der AEMP mit sehr kleinen Codes ausgestattet sind. Diese müssen zudem auf einer mittleren Entfernung (1 bis 2~m) gelesen werden können.
%
\insertMore{Datamatrix Codes}