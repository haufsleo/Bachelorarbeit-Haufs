\chapter{Konzept}
% 0,5 Seiten
Im Folgenden soll ein Konzept vorgestellt werden, wie Smartglasses in der Sterilgutversorgung eingesetzt werden können. Dazu werden zunächst die Interaktionsmöglichkeiten erläutert, dann die Möglichkeiten der Objekterkennung dargelegt und dies mit möglichen Peripheriegeräten konkretisiert. Eine Beschreibung der Medienverwaltung, -präsentation sowie -erfassung und -erstellung werden abschließend beschrieben.
%
% - - - - - - - - - - - - - - - - - - - - - - - - 
%
\section{Interaktionsmöglichkeiten}
% 2 Seiten
Wie in Kapitel \ref{sec:Stand_der_Technik} bereits beschrieben gibt es bei Smartglasses verschiedene Arten der Interaktion, und damit der Eingabe an der Brille. Je nach Hardware und aufgespielter Software (App) sind verschiedene Interaktionsmöglichkeiten denkbar. 

So kann die Brille in allen in Kapitel \ref{sec:VergleichSmartglasses} beschriebenen Geräten per Sprachsteuerung bedient werden. Dies ist besonders im Arbeitsumfeld der Medizinprodukteaufbereitung wichtig da, wie bereits ausführlich in Kapitel \ref{sec:Ablauf_der_Medizinprodukteaufbereitung} und \ref{sec:Spezifische_Anforderungen_an_Smartglasses} beschrieben, die Hände der Angestellten frei sein müssen.

Manche Brillen wie die \emph{Glass} von Google besitzen zusätzlich eine Touchbar, die eine Berührungssensitive Steuerung ermöglichen. Denkbar sind hier bei entsprechender Technik auch evtl. Touch- Gestensteuerung wie \enquote{Pinch-to-zoom}.

Durch die Kamera der Smartglass, die auch alle in Kapitel \ref{sec:VergleichSmartglasses} beschriebenen Geräte hatten, lassen sich zudem Gesten- und Kopfbewegungen erkennen. So lassen sich über Gesten der Hände beispielsweise mögliche virtuelle Elemente in der grafischen Oberfläche der Brille auswählen. Swipe- und Scrollgesten wären so ebenfalls denkbar wie die zuvor erwähnten Pinch-to-zoom Geste.

Per Barcodes und QR-Codes können direkte Befehle an die Brille gegeben werden. So ist es möglich Methoden der App auf der Brille aufzurufen oder Hyperlinks auszuwählen. Es wäre beispielsweise denkbar einen QR-Code zu verwenden, um Videos zu einem Bauteil anzuzeigen. Ein Barcode kann als Beschriftung eines Siebes oder Sterilgutes verwendet werden und damit Automatisch getracked werden. Barcodes können lediglich Zahlen speichern und sind in ihrer Anzahl an Ziffern stark begrenzt. QR-Codes sind zweidimensionale (Matrix-) Barcodes  und können bis zu 4.296 Alphanumerische Zeichen speichern \cite{INCORPORATED2018}. So können auch komplexe Informationen \emph{mit einem Blick} erfasst werden.

Die im folgenden Unterkapitel (\ref{sec:Objekterkennung}) näher beschriebene Objekterkennung ermöglicht ebenfalls starke Eingabemöglichkeiten.

%
% - - - - - - - - - - - - - - - - - - - - - - - - 
%
\section{Objekterkennung}
\label{sec:Objekterkennung}
% 2 Seiten 
Objekterkennung ist in den letzten Jahren massiv weiterentwickelt worden. Firmen wie Apple \cite{Apple2018b}, Microsoft \cite{Girshick2015} oder Facebook \cite{Schroepfer2015} haben große Fortschritte in der Technologie der Objekterkennung gemacht. Diese großen Meilensteine konnten erreicht werden durch die großen Fortschritte in der Künstlichen Intelligenz mit neuronalen Netzwerken \cite{Apple2018b, Schroepfer2015}. Zur modernen Objekterkennung mittels Machine Learning- Algorithmen ist es heute möglich in sehr kurzer Zeit Objekte selbst in komplizierten Umgebungen zu erkennen \cite{Schroepfer2015}. \todo{..?}


%
% - - - - - - - - - - - - - - - - - - - - - - - - 
%
\subsection{Externe Peripherie}
\label{sec:Externe_Peripherie}
% 1 Seite 
%
% - - - - - - - - - - - - - - - - - - - - - - - - 
%
\subsection{Medienverwaltung}
\label{sec:Medienverwaltung}
% 2 Seiten 
%
% - - - - - - - - - - - - - - - - - - - - - - - - 
%
\section{Medienpräsentation}
\label{sec:Medienpraesentation}
% 2 Seiten
%
% - - - - - - - - - - - - - - - - - - - - - - - - 
%
\section{Medienerfassung/-erstellung}
\label{sec:Medienerfassung_-erstellung}
% 2 Seiten