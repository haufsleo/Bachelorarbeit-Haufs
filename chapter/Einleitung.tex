%
%
%
% - - - - - Einleitung - - - - - - - - 
%
%
%
\chapter{Einleitung}
\label{ch:Einleitung}
% 2 Seiten
Augmented Reality-Geräte gehören laut einer von Gartner beauftragten Analyse \cite{Linden2003} zu den Top 10 Technologien für das Jahr 2018 \cite{Panetta2017a}. Es wird ein fundamentaler Wechsel in der Art und Weise erwartet, wie Nutzer mit der digitalen Welt interagieren. 

Ebenso wie Augmented Reality gehören die sogenannten \emph{wearables}, also am Körper getragene Computer, zu den Geräten mit den höchsten Erwartungen im Bereich des Marktpotentials aus dem Jahre 2015 \cite{Levy2015}. Neben Smartwatches, die sich im privaten Bereich bereits durchgesetzt haben, werden Datenbrillen (Smartglasses, Data Glasses, Head Mounted Displays (HMD), Head-Up Displays (HUD) oder auch Head Worn Displays \cite{Zobel2016}) neben dem privaten auch im beruflichen Umfeld eingesetzt. Smartglasses sind eine der am intensivsten und vielversprechendsten diskutierten Technologien im professionellen Umfeld \cite{Hein2016}. Die Mobilität von Computern nahm mit der Erfindung von Smartphones rasant zu. \emph{Wearable Technologies} beschleunigen diesen Effekt weiter. Computer gehören damit zum allgegenwärtigen Bestandteil des Lebens vieler Anwender und begleiten Nutzer den gesamten (Berufs-) Alltag hinweg. Wearables stellen eine völlig neuartige Schnittstelle zwischen Mensch und Computer dar \cite[S.~25f]{Schwenke2016}. Bei der Veröffentlichung kürte die Zeitschrift Time Magazine die Erfindung Google Glass als eine der \enquote{Best Inventions of the Year} \cite{Bilton2015}.

Smartglasses sind Brillen, die es ermöglichen, digitale Informationen direkt im realen Umfeld anzuzeigen. So haben Nutzer von Smartglasses die Informationen immer im Blickfeld, ohne auf ein Gerät wie ein Smartphone schauen zu müssen \cite{Due2014Glasses}.

Diese Interaktion mittels Augmented Reality-Wearables wie Smartglasses wird im beruflichen Umfeld bereits erfolgreich im Bereich der Logistikunternehmen eingesetzt \cite{Plutz}.
Ebenso werden sie in Museen als interaktiver Guide verwendet \cite{Hein2016}. 
Wearables bieten hier die Möglichkeit im Arbeitsalltag, in dem beispielsweise die Hände frei sein müssen, kontextbezogene Informationen bereitzustellen und so die Arbeit zu erleichtern \cite{Zobel2016}. Smartglasses ermöglichen es, kontextsensitive Informationen darzustellen und Objekte zu erkennen und zu klassifizieren.
%

Im Rahmen dieser Arbeit wird der Einsatz von Smartglasses im Bereich der medizinischen Sterilgutversorgung untersucht. 

Für Operationen werden chirurgische Instrumente benötigt, die steril sein müssen, sogenannte Sterilgüter.
Medizinische Instrumente in Krankenhäusern werden zum Teil nicht nur immer kleiner, sondern auch immer teurer, sodass eine Wiederverwendung mancher Instrumente elementar wichtig ist. Dies geschieht durch sogenannte Sterilisation der Geräte, die in der Regel in der Medizinprodukteaufbereitung durchgeführt wird, die Teil der Zentralen Sterilgutversorgungsabteilung (ZSVA), neuerdings auch Aufbereitungseinheit für Medizinprodukte (AEMP) genannt, von Krankenhäusern ist. In der AEMP werden medizinische Geräte gereinigt, desinfiziert und sterilisiert und somit unter strengen rechtlichen und hygienischen Vorgaben wiederaufbereitet \cite{AKI-ArbeitskreisInstrumenten-Aufbereitung2012}.

Wie wichtig eine gut funktionierende Sterilgutversorgung ist, zeigen Problemfälle wie im Uniklinikum Mannheim \cite{Brandt2015} aus dem Jahr 2015 oder im Klinikum Fulda im Jahr 2012 \cite{HygieneFuldar2012}, bei denen nicht nur ein starker menschlicher Schaden, sondern für die betroffenen Krankenhäuser auch ein millionenschwerer Imageschaden durch mangelhaft sterilisierte chirurgische Instrumente entstanden ist. Eine reibungslos funktionierende Sterilgutversorgung stellt also ein Kernthema der Patientensicherheit dar und ist gleichzeitig mit hohem Kostendruck verbunden. Dieser führt oftmals zu Personaleinsparungen, was den Druck auf die Mitarbeiter nochmals verschärft. Die AEMP steht ebenfalls unter einer enormen Belastung, da bedingt durch den hohen Preis die Anzahl der Geräte und chirurgischen Instrumente möglichst klein gehalten werden muss. Chirurgische Instrumente werden zudem immer komplexer und somit die Anforderungen an die Angestellten der AEMP immer höher. Durch die hohe Komplexität der Geräte sind oft nur einzelne Angestellte in der Lage, besonders komplizierte Instrumente aufzubereiten. Ziel einer technischen Unterstützung innerhalb der AEMP ist, die Arbeit der Angestellten zu erleichtern und möglichst viele Personen in die Lage zu versetzen, bislang unbekannte Instrumente aufzuarbeiten. Es wird immer wichtiger, die Angestellten der AEMP auch durch technischer Hilfsmittel zu unterstützen. Die Unterstützung mittels Augmented Reality über Smartglasses kann möglicherweise ein gutes Mittel sein, um im Feld der aufgrund der hoch infektiösen Keime steril gehaltenen Atmosphäre der Medizinprodukteaufbereitung eingesetzt zu werden. Die Brillen ermöglichen hier eine freihändige Bedienung und Bereitstellung von Informationen. Die zunehmende Komplexität und Individualisierung medizinischer Verfahren macht die korrekte Handhabung und Sterilisierung der Instrumente zu einer besonderen Herausforderung für das Personal. Mittels der Datenbrillen können auf jedes medizinische Instrument angepasste Informationen dargestellt werden. 

Im Rahmen dieser Arbeit wird der Einsatz von Smartglasses in der AEMP analysiert, die mögliche Unterstützung auf Grundlage multimedialer Anwendungen erörtert und mittels Experteninterviews evaluiert. 